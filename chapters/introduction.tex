\section{Introduction}
Software product lines (SPLs) represent an approach to software engineering, facilitating the systematic and efficient management of variations within a family of related software systems. 
SPLs permit the development of a set of features, from which a multitude of product variants can be constructed, thereby addressing the requirements of customers or market segments.
However, this flexibility introduces significant complexity, particularly with regard to testing and verifying the correctness of individual product configurations.

In the context of software product lines, the term \textit{sampling} is used to describe a process through which a subset of the total population of items is selected for further analysis.
Uniform random sampling is of particular importance for unbiased testing, analysis, and quality assurance \cite{plazar-spl-sampling, dutra-sat-sampling, oh-spl-random-sampling, oh-t-wise}, as it ensures that each product variant has an equal chance of being selected, thereby avoiding biases that might lead to incorrect conclusions about the overall product line.

The potential of quantum computing to solve complex problems more efficiently than classical approaches offers promising solutions for many computational challenges, including those found in SPLs.
In this context, Grover's Quantum Algorithm represents a powerful tool for searching unsorted databases or solving satisfiability problems \cite{math11081888, Hangleiter_2023}.
When applied to a Boolean formula representing constraints in an SPL, Grover's algorithm can be employed to identify satisfying configurations with a quadratic speedup compared to brute forcing.
This ability makes it a strong candidate for enabling uniform random sampling in SPLs, as it can efficiently identify valid product configurations from a large space.

Although there already are existing approaches for a quantum implementation, they frequently allow for optimization.
Previous research on the application of Grover's algorithm to SPLs did not fully optimize the quantum circuit, resulting in inefficiencies in gate operations and qubit utilization.
Specifically, the prior implementation failed to remove obsolete gates, such as pairs of X gates that cancel each other out, and did not adequately identify opportunities for qubit reuse, where qubits performing the same function could have been shared rather than redundantly created.
These oversights not only increase the complexity of the quantum circuit but also reduce the overall efficiency and practicality of the quantum algorithm.

This paper builds on the foundational work of previous research \cite{ammermann2023quantumcomputingimproveuniform} by introducing significant improvements to the quantum circuit design for uniform random sampling in SPLs.
By optimizing the quantum gate sequence and implementing effective qubit reuse strategies, we aim to reduce the resource requirements and enhance the performance of Grover's algorithm in this context.
Our contributions not only streamline the quantum computation process but also make the application of quantum algorithms to SPLs more feasible and effective, paving the way for more advanced and efficient quantum software engineering practices.